\documentclass{article}
\usepackage{geometry}
\usepackage{amsmath}
\usepackage{amssymb}
\usepackage{tikz-cd}
\usepackage{cmll}

\newtheorem{definition}{Definition}
\newtheorem{proposition}{Proposition}
\newtheorem{question}{Question}

\title{Game Interpretation of Dialectica constructions}

\author{
  Colin Bloomfield \and
  J\'er\'emie Koenig \and
  Valeria de Paiva \and
  Jan Rooduijn}

\begin{document}

\maketitle

\section{Coherence Games}

Coherence spaces provide a general and uniform model of interaction.
Like Dialectica categories they constitute a model of linear logic,
and the constructions on coherence spaces
can be understood in terms of the usual game interpretation of linear logic.
Hence they are a convenient setting into which
Dialectica games can be interpreted
in order to formally understand their operational interpretation.

However,
standard coherence spaces do not provide a way to
express winning conditions.
To remedy this, we relax the reflexivity condition
of coherence relations.
We will then understand a self-related token $a \coh a$
as a winning play;
when $a \coh a$ does not hold,
the token $a$ will represent a possible but losing play.

\begin{definition}[Coherence Game]
A coherence game $A = \langle |A|, {\coh_A} \rangle$
is a set $|A|$ of \emph{tokens}
together with a symmetric relation $\coh_A$.
\end{definition}

\begin{definition}[Morphism]
A morphism $f : A \rightarrow B$
between the coherence games $A$ and $B$
is a relation $|f| \subseteq |A| \times |B|$
such that
when $a \mathrel{[f]} b$ and $a' \mathrel{[f]} b'$,
we have $a \coh_A a' \Rightarrow b \coh_B b'$:
\[
  \begin{tikzcd}
    a \ar[r, dash, "f"] \ar[d, dash, "\coh_A"'] &
    b \ar[d, dash, dotted, "\coh_B"] \\
    a' \ar[r, dash, "f"'] & b'
  \end{tikzcd}
\]
The morphism $\mathrm{id}_A : A \rightarrow A$
is the relation $=_{|A|}$.
Morphisms compose as relations.
\end{definition}

\begin{definition}[Winning clique]
A winning clique in a coherence game $A$
is a subset $\sigma \subseteq |A|$
such that
$\forall a, a' \in \sigma \mathbin{.} a \coh_A a'$.
\end{definition}

\begin{definition}[Multiplicatives]
For the coherence games $A, B$
we define the coherence games
$A \otimes B$, $A \rhd B$, $A \parr B$, $A \multimap B$
on the set of tokens
\[
  |A \otimes B| =
  |A \parr B| =
  |A \multimap B| =
  |A \rhd B| =
  |A| \times |B|
  \,.
\]
with the coherence relations given by:
\begin{align*}
  (a, a') \coh_{A \otimes B} (b, b') &\::\Leftrightarrow\:
    a \coh_A a' \wedge b \coh_B b' \\
  (a, a') \coh_{A \parr B} (b, b') &\::\Leftrightarrow\:
    a \coh_A a' \vee b \coh_B b' \\
  (a, a') \coh_{A \rhd B} (b, b') &\::\Leftrightarrow\:
    a \coh_A a' \wedge (a = a' \Rightarrow b \coh_B b') \\
  (a, a') \coh_{A \multimap B} (b, b') &\::\Leftrightarrow\:
    a \coh_A a' \Rightarrow b \coh_B b'
\end{align*}
\end{definition}

Note that the winning cliques of $A \multimap B$
are the morphisms from $A$ to $B$.
It is easy to verify that
$A \otimes B \rightarrow C \cong A \rightarrow B \multimap C$.

\begin{definition}
The \emph{dual} $A^\bot$ of a coherence game $A$
is defined by:
\[
  |A^\bot| := |A|
  \qquad
  a \coh_{A^\bot} a' :\Leftrightarrow
	  \lnot (a \coh_A a')
\]
\end{definition}

Note that $A \multimap B = A^\bot \parr B$.
As usual, $\otimes$ and $\parr$ are dual with each other.
However,
while $\rhd$ is self-dual in coherence spaces,
adding the winning condition breaks this property:
the system must win both games to win $A \rhd B$,
but the environment only has to win one of them
in order to win the overall game.

\section{Embedding $\Sigma \Pi \mathbf{2}$}

We use the notations used in Nelson's write-up.
Since $D\mathbf{Set} \subseteq \Sigma \Pi \mathbf{2}$,
the functor we define also applies to it.

Consider $(p, P) \in \Sigma \Pi \mathbf{2}$,
that is $p \in \mathbf{Poly}$
and $P \subseteq \sum_{i \in p(1)} p[i]$.
The corresponding coherence game
$E(p, P)$
is defined as follows:
\begin{align*}
  |E (p, P)| \quad &:= \quad \sum_{i \in p(1)} p[i] \\
  (i, a) \coh_{E (p, P)} (j, b) \quad &:\Leftrightarrow\quad
    i = j \wedge (a = b \Rightarrow P(i, a))
\end{align*}
In addition,
given a morphism
\[ (f, F) : (p, P) \rightarrow (q, Q) \in \Sigma\Pi\mathbf{2}\,, \]
so that $f : p(1) \rightarrow q(1)$
and $F \in \prod_{i \in p(1)} p[i]^{q[f(i)]}$
such that $P(i, F_i(b)) \Rightarrow Q(f(i), b)$,
we define the corresponding morphism
$E (f, F) : E (p, P) \rightarrow E (q, Q) \in \mathbf{CoG}$
in the following way:
\[
  (i, a) \mathrel{[E (f, F)]} (j, b)
    \::\Leftrightarrow\:
    f(i) = j \wedge a = F_i(b)
\]

\begin{proposition}
These constructions define a functor
$E : \Sigma\Pi\mathbf{2} \rightarrow \mathbf{CoG}$.
\end{proposition}

Since
$G\mathbf{Set} \subseteq D\mathbf{Set} \subseteq \Sigma\Pi\mathbf{2}$,
the functor applies to these categories as well.

\begin{proposition}
The functor $E$ is faithful.
\end{proposition}

However,
there are many more morphisms in $\mathbf{CoG}(E A, E B)$
than in $\Sigma\Pi\mathbf{2}(A, B)$,
since $\mathbf{CoG}$ contains partially defined morphisms
such as $\bot$ in addition to those embedded from $\mathbf{CoG}$.
It may possible to obtain a full functor
from something such as $D \mathbf{C}$ construction
performed on a category of partial functions.

\begin{question}
Does that work?
\end{question}

\begin{question}
Conversely,
can we restrict $\mathbf{CoG}$ to maximal morphisms
to obtain the same result?
Is $D\mathbf{C}$ or (more plausibly) $G\mathbf{C}$
equivalent to a full subcategory of that?
\end{question}

\begin{question}
How do the monoidal structure on
$D\mathbf{C}$,
$G\mathbf{C}$ and
$\Sigma\Pi\mathbf{2}$
carry through $E$?
\end{question}

\end{document}
